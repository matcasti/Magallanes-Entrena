\documentclass{article}
\usepackage{array}
\usepackage{caption}
\usepackage{graphicx}
\usepackage{siunitx}
\usepackage{ulem}
\usepackage{colortbl}
\usepackage{multirow}
\usepackage{hhline}
\usepackage{calc}
\usepackage{tabularx}
\usepackage{threeparttable}
\usepackage{wrapfig}
\usepackage{adjustbox}
\usepackage{hyperref}
% These are LaTeX packages. You can install them using your LaTex management software,
% or by running `huxtable::install_latex_dependencies()` from within R.
% Other packages may be required if you use non-standard tabulars (e.g. tabulary).
\pagenumbering{gobble}

\begin{document}


  \providecommand{\huxb}[2]{\arrayrulecolor[RGB]{#1}\global\arrayrulewidth=#2pt}
  \providecommand{\huxvb}[2]{\color[RGB]{#1}\vrule width #2pt}
  \providecommand{\huxtpad}[1]{\rule{0pt}{#1}}
  \providecommand{\huxbpad}[1]{\rule[-#1]{0pt}{#1}}

\begin{table}[ht]
\begin{centerbox}
\begin{threeparttable}
 \setlength{\tabcolsep}{0pt}
\begin{tabular}{l l}


\hhline{}
\arrayrulecolor{black}

\multicolumn{1}{!{\huxvb{0, 0, 0}{0}}l!{\huxvb{0, 0, 0}{0}}}{\huxtpad{6pt + 1em}\raggedright \hspace{6pt} Dimensión \hspace{6pt}\huxbpad{6pt}} &
\multicolumn{1}{l!{\huxvb{0, 0, 0}{0}}}{\huxtpad{6pt + 1em}\raggedright \hspace{6pt} Significado \hspace{6pt}\huxbpad{6pt}} \tabularnewline[-0.5pt]


\hhline{}
\arrayrulecolor{black}

\multicolumn{1}{!{\huxvb{0, 0, 0}{0}}l!{\huxvb{0, 0, 0}{0}}}{\huxtpad{6pt + 1em}\raggedright \hspace{6pt} Función física \hspace{6pt}\huxbpad{6pt}} &
\multicolumn{1}{l!{\huxvb{0, 0, 0}{0}}}{\huxtpad{6pt + 1em}\raggedright \hspace{6pt} Grado en el que la falta de salud limita las actividades físicas de la vida diaria, como el cuidado personal, caminar, subir escaleras, coger o transportar cargas, y realizar esfuerzos moderados e intensos. \hspace{6pt}\huxbpad{6pt}} \tabularnewline[-0.5pt]


\hhline{}
\arrayrulecolor{black}

\multicolumn{1}{!{\huxvb{0, 0, 0}{0}}l!{\huxvb{0, 0, 0}{0}}}{\huxtpad{6pt + 1em}\raggedright \hspace{6pt} Rol físico \hspace{6pt}\huxbpad{6pt}} &
\multicolumn{1}{l!{\huxvb{0, 0, 0}{0}}}{\huxtpad{6pt + 1em}\raggedright \hspace{6pt} Grado en el que la falta de salud interfiere en el trabajo y otras actividades diarias, produciendo como consecuencia un rendimiento menor del deseado, o limitando el tipo de actividades que se puede realizar o la dificultad de las mismas. \hspace{6pt}\huxbpad{6pt}} \tabularnewline[-0.5pt]


\hhline{}
\arrayrulecolor{black}

\multicolumn{1}{!{\huxvb{0, 0, 0}{0}}l!{\huxvb{0, 0, 0}{0}}}{\huxtpad{6pt + 1em}\raggedright \hspace{6pt} Dolor corporal \hspace{6pt}\huxbpad{6pt}} &
\multicolumn{1}{l!{\huxvb{0, 0, 0}{0}}}{\huxtpad{6pt + 1em}\raggedright \hspace{6pt} Medida de la intensidad del dolor padecido y su efecto en el trabajo habitual y en las actividades del hogar. \hspace{6pt}\huxbpad{6pt}} \tabularnewline[-0.5pt]


\hhline{}
\arrayrulecolor{black}

\multicolumn{1}{!{\huxvb{0, 0, 0}{0}}l!{\huxvb{0, 0, 0}{0}}}{\huxtpad{6pt + 1em}\raggedright \hspace{6pt} Salud general \hspace{6pt}\huxbpad{6pt}} &
\multicolumn{1}{l!{\huxvb{0, 0, 0}{0}}}{\huxtpad{6pt + 1em}\raggedright \hspace{6pt} Valoración personal del estado de salud, que incluye la situación actual y las perspectivas futuras y la resistencia a enfermar. \hspace{6pt}\huxbpad{6pt}} \tabularnewline[-0.5pt]


\hhline{}
\arrayrulecolor{black}

\multicolumn{1}{!{\huxvb{0, 0, 0}{0}}l!{\huxvb{0, 0, 0}{0}}}{\huxtpad{6pt + 1em}\raggedright \hspace{6pt} Vitalidad \hspace{6pt}\huxbpad{6pt}} &
\multicolumn{1}{l!{\huxvb{0, 0, 0}{0}}}{\huxtpad{6pt + 1em}\raggedright \hspace{6pt} Sentimiento de energía y vitalidad, frente al de cansancio y desánimo. \hspace{6pt}\huxbpad{6pt}} \tabularnewline[-0.5pt]


\hhline{}
\arrayrulecolor{black}

\multicolumn{1}{!{\huxvb{0, 0, 0}{0}}l!{\huxvb{0, 0, 0}{0}}}{\huxtpad{6pt + 1em}\raggedright \hspace{6pt} Función social \hspace{6pt}\huxbpad{6pt}} &
\multicolumn{1}{l!{\huxvb{0, 0, 0}{0}}}{\huxtpad{6pt + 1em}\raggedright \hspace{6pt} Grado en el que los problemas físicos o emocionales derivados de la falta de salud interfieren en la vida social habitual. \hspace{6pt}\huxbpad{6pt}} \tabularnewline[-0.5pt]


\hhline{}
\arrayrulecolor{black}

\multicolumn{1}{!{\huxvb{0, 0, 0}{0}}l!{\huxvb{0, 0, 0}{0}}}{\huxtpad{6pt + 1em}\raggedright \hspace{6pt} Rol emocional \hspace{6pt}\huxbpad{6pt}} &
\multicolumn{1}{l!{\huxvb{0, 0, 0}{0}}}{\huxtpad{6pt + 1em}\raggedright \hspace{6pt} Grado en el que los problemas emocionales afectan al trabajo y otras actividades diarias, considerando la reducción del tiempo dedicado, disminución del rendimiento y del esmero en el trabajo \hspace{6pt}\huxbpad{6pt}} \tabularnewline[-0.5pt]


\hhline{}
\arrayrulecolor{black}

\multicolumn{1}{!{\huxvb{0, 0, 0}{0}}l!{\huxvb{0, 0, 0}{0}}}{\huxtpad{6pt + 1em}\raggedright \hspace{6pt} Salud mental \hspace{6pt}\huxbpad{6pt}} &
\multicolumn{1}{l!{\huxvb{0, 0, 0}{0}}}{\huxtpad{6pt + 1em}\raggedright \hspace{6pt} Valoración de la salud mental general, considerando la depresión, ansiedad, autocontrol, y bienestar general. \hspace{6pt}\huxbpad{6pt}} \tabularnewline[-0.5pt]


\hhline{}
\arrayrulecolor{black}
\end{tabular}
\end{threeparttable}\par\end{centerbox}

\end{table}



\end{document}